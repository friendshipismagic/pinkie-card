\documentclass[a4paper,12pt]{article}

%% Language and font encodings
\usepackage[english]{babel}
\usepackage[utf8x]{inputenc}
\usepackage[T1]{fontenc}

%% Sets page size and margins
\usepackage[a4paper,top=3cm,bottom=2cm,left=3cm,right=3cm,marginparwidth=1.75cm]{geometry}

%% Useful packages
\usepackage{amsmath}
\usepackage{graphicx}

\title{Compte rendu projet IGR203}
\author{Alexis Bauvin -- Clément Decoodt -- Ronan Desplanques -- Ming Yang}

\begin{document}
\maketitle

\section{Le design}

\subsection{Les pistes explorées}

\begin{itemize}
\item Design adapté à une tablette :

La première piste de design est adaptée à un écran de tablette et utilise la taille de ce support pour proposer une interface originale et agréable en plus d'être fonctionnelle. Il est fait usage de métaphores qui rappellent les cartes de restaurant traditionnelles et la partie de l'interface qui exprime l'état courant de la commande a la forme d'une table de restaurant où les éléments choisis sont disposés.

\item Design adapté à un téléphone :

Cette idée de design correspond à une implémentation sur téléphone. Sur un écran plutôt petit elle privilégie la simplicité et la clarté en utilisant une interface de listes. Les boutons sont grands et occupent tout l'écran, il n'y a pas d'éléments de décoration.

\item Design traditionnel :

Ce design est fondé sur l'idée d'émuler au maximum une carte de menu classique. La partie principale de l'interface prend la forme d'un menu traditionnel, et ce n'est qu'en touchant un élément qu'on peut accéder à des informations spécifiques et commander. Pour accentuer l'aspect "papier" de l'interface, on reprend des caractéristiques des liseuses électroniques comme les animations de tournage de page et la teinte du fond d'écran.

\end{itemize}

\subsection{Le design choisi}

Le premier design proposé a été choisi pour l'interface principale du système. C'est celui qui permet d'apporter le plus de valeur ajoutée par rapport au système classique, et les contraintes techniques de son implémentation ne paraissent pas insurmontables. Les besoins de personnalisation et d'interactivité de l'interface sont bien remplis par le design. Le design est intuitif et ne nécessite pas de phase d'apprentissage pour être fonctionnel ; le  principe de "recognition over recall" est respecté.

Les objectifs les moins bien remplis par ce design sont les exigences de portabilité : pour être correctement utilisée l'interface doit être disposée sur un écran avec une résolution suffisante ; la taille de l'écran doit également être plutôt grande, c'est-à-dire au moins correspondre à un format tablette. Ce design n'est pas adapté au déploiement d'une application pour téléphone que les clients pourraient installer sur leurs propres appareils.

\subsection{Évolution des besoins lors de la conception du design}



\end{document}

