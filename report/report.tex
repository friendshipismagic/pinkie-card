\documentclass[a4paper,12pt]{article}

%% Language and font encodings
\usepackage[english]{babel}
\usepackage[utf8x]{inputenc}
\usepackage[T1]{fontenc}

%% Sets page size and margins
\usepackage[a4paper,top=3cm,bottom=2cm,left=3cm,right=3cm,marginparwidth=1.75cm]{geometry}

%% Useful packages
\usepackage{amsmath}
\usepackage{graphicx}

\title{Compte rendu projet IGR203}
\author{Alexis Bauvin -- Clément Decoodt -- Ronan Desplanques -- Ming Yang}

\begin{document}
\maketitle

\section{Le design}

\subsection{Les pistes explorées}

\begin{itemize}
\item Design adapté à une tablette :

La première piste de design est adaptée à un écran de tablette et utilise la taille de ce support pour proposer une interface originale et agréable en plus d'être fonctionnelle. Il est fait usage de métaphores qui rappellent les cartes de restaurant traditionnelles et la partie de l'interface qui exprime l'état courant de la commande a la forme d'une table de restaurant où les éléments choisis sont disposés.

\item Design adapté à un téléphone :

Cette idée de design correspond à une implémentation sur téléphone. Sur un écran plutôt petit elle privilégie la simplicité et la clarté en utilisant une interface de listes. Les boutons sont grands et occupent tout l'écran, il n'y a pas d'éléments de décoration.

\end{itemize}

\end{document}

